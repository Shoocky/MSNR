% !TEX encoding = UTF-8 Unicode

\documentclass[a4paper]{article}

\usepackage{color}
\usepackage{url}
\usepackage[T2A]{fontenc} % enable Cyrillic fonts
\usepackage[utf8]{inputenc} % make weird characters work
\usepackage{graphicx}
\usepackage{listings}
\usepackage{amsthm}
\usepackage[english,serbian]{babel}
%\usepackage[english,serbianc]{babel} %ukljuciti babel sa ovim opcijama, umesto gornjim, ukoliko se koristi cirilica

\usepackage[unicode]{hyperref}
\hypersetup{colorlinks,citecolor=green,filecolor=green,linkcolor=blue,urlcolor=blue}

\lstMakeShortInline[columns=fixed]|

%\newtheorem{primer}{Пример}[section] %ćirilični primer
\newtheorem{primer}{Primer}[section]
\newtheorem{definicija}{Definicija}[section]

\begin{document}

\title{Git kroz primere\\ \small{Seminarski rad u okviru kursa\\Metodologija stručnog i naučnog rada\\ Matematički fakultet}}

\author{Nemanja Radosavljević, Milan Stojković\\ shoocky1337@gmail.com, milanzp@hotmail.com}
\date{9.~april 2016.}
\maketitle

\abstract{

U ovom tekstu je ukratko prikazana osnovna forma seminarskog rada. Obratite pažnju da je pored ove .pdf datoteke, u prilogu i odgovarajuća .tex datoteka, kao i .bib datoteka korišćena za generisanje literature. Na prvoj strani seminarskog rada su naslov, apstrakt i sadržaj, i to sve mora da stane na prvu stranu! Kako bi Vaš seminarski zadovoljio standarde i očekivanja, koristite uputstva i materijale sa predavanja na temu pisanja seminarskih radova. Ovo je samo šablon koji se odnosi na fizički izgled seminarskog rada (šablon koji \emph{morate} da ispoštujete!) kao i par tehničkih pomoćnih uputstava. Molim Vas da kada budete predavali seminarski rad, imenujete datoteke tako da sadrže temu seminarskog rada, kao i imena i prezimena članova grupe (ili samo temu i prezimena, ukoliko je sa imenima predugačko). Predaja seminarskih radova biće isključivo preko web forme, a NE slanjem mejla.

\tableofcontents

\newpage

\section{Uvod}
\label{sec:uvod}

Ко жели, може да пише рад ћирилицом. У том случају, неопходно је да су инсталирани одговарајући пакети: texlive-fonts-extra, texlive-latex-extra, texlive-lang-cyrillic, texlive-lang-other. \\

Uz sve novouvedene termine u zagradi naglasiti od koje engleske reči termin potiče. Naredni primeri ilustruju način uvođenja enlegskih termina kao i citiranje.

\begin{primer}
Problem zaustavljanja (eng.~{\em halting problem}) je neodlučiv \cite{haltingproblem}.
\end{primer}

\begin{primer}
Za prevođenje programa napisanih u programskom jeziku C može se koristiti GCC kompajler \cite{gcc}.
\end{primer}

\begin{primer}
 Da bi se ispitivala ispravost softvera, najpre je potrebno precizno definisati njegovo ponašanje \cite{laski2009software}. 
\end{primer}

Reference koje se koriste u ovom tekstu zadate su u datoteci {\em seminarski.bib}. Prevođenje u pdf format u Linux okruženju može se uraditi na sledeći način:
\begin{verbatim}
pdflatex TemaImePrezime.tex 
bibtex TemaImePrezime.aux 
pdflatex TemaImePrezime.tex 
pdflatex TemaImePrezime.tex 
\end{verbatim}
Prvo latexovanje je neophodno da bi se generisao {\em .aux} fajl. {\em bibtex} proizvodi odgovarajući {\em .bbl} fajl koji se koristi za generisanje literature. 
Potrebna su dva prolaza (dva puta pdflatex) da bi se reference ubacile u tekst (tj da ne bi ostali znakovi pitanja umesto referenci). Dodavanjem novih referenci potrebno je ponoviti ceo postupak.  


Broj naslova i podnaslova je proizvoljan. Neophodni su samo Uvod i Zaključak. Na poglavlja unutar teksta referisati se po potrebi. 
\begin{primer}
U odeljku \ref{sec:naslov1} precizirani su osnovni pojmovi, dok su zaključci dati u odeljku \ref{sec:zakljucak}.
\end{primer}

Još jednom da napomenem da nema razloga da pišete:
\begin{verbatim}
\v{s} i \v{c} i \'c ...
\end{verbatim}
Možete koristiti srpska slova
\begin{verbatim}
š i č i ć ... 
\end{verbatim}


Ovde pišem uvodni tekst.
Ovde pišem uvodni tekst. 
Ovde pišem uvodni tekst. 
Ovde pišem uvodni tekst. 


\section{Sekcije git projekta i životni ciklus datoteke}
\label{zivotni_ciklus}
Pre nego što počnemo sa razmatranjem korišćenja git sistema bilo bi zgodno uvesti neki osnovne pojmove vezane za kontrolu verzija. Ovo može koristiti korisnicima koji nemaju nikakvog prethodnog iskustva sa sistemima za kontrolu verzija. Sa druge strane, korisnici koji imaju iskustva mogu preskočiti ovaj deo i nastaviti od "Sekcije git projekta". Terminologija varira od sistema do sistema, pa ćemo se fokusariti samo na one univerzalne termine, koji su zajednički za sve sisteme.

\begin{definicija}
Repozitorijum ili baza predstavlja skup svih fajlava koja se nalaze pod kontrolom verzije.
\end{definicija}

\begin{definicija}
Verzija predstavlja sliku stanja sadržaja repozitorijuma u nekom trenutku razvoja.
\end{definicija}

\begin{definicija}
Radna kopija (working copy) predstavlja lokalnu kopiju verzije koja se menja tokom rada.
\end{definicija}

\begin{definicija}
Podizanje promene ili komit (commit) predstavlja akciju kojom se trenutno stanje zapisuje u repozitorijum. Takodje može da se koristi za referisanje odredjene promene.
\end{definicija}

\begin{definicija}
Grana (branch) predstavlja imenovan tok razvoja nezavistan od ostalih tokova.
\end{definicija}
Iako su grane medjusobno nezavisne, one uglavnom imaju zajednički neki deo njihove istorije. Svaka grana pocinje kao kopija neke druge granaei odatle nastavlja da se razvija nezavisno. Početna grana, odnosno ona koja nije nastala kopijom neke druge se sreće pod nazivima trunk, mainline, master.

\begin{definicija}
Konflikt predstavlja situaciju u kojoj dva korisnika pokusavaju da podignu izmenu koja se odnosi na isti deo iste datoteke.
\end{definicija}
Više reči o konfliktima će biti u poglavlju *.

\subsection{Sekcije git projekta}
\label{subsec:sekcije}

\subsection{Životni ciklus datoteke}
\label{subsec:ciklus}




\section{Kreiranje repozitorijuma i osnovne komande}
\label{sec:kreiranje}

Kako bi Git mogao da prati ko je napravio izmene, pre svega je potrebno da mu se predstavite. Izmedju ostalog, za to koristimo  \texttt{git config} komandu.
\linebreak


\begin{lstlisting}[language=bash]
git config --global user.name <name>
git config --global user.email <email>

\end{lstlisting}

Podešavanje imena korisnika je obavezno, jer se svaki komit vezuje za korisnika koji ga je napravio. Opcija |--global| označava da je podešavanje globalnog opsega, tj. važi za sve repozitorijume u sistemu. Izostavljanjem ove opcije podešavanje se vezaju samo za tekući repozitorijum. Sva podešavanja sa globnim opsegom se čuvaju u \textit{.gitconfig} datoteci u \textit{home} direktorijumu na Linux sistemima. Direktnim menjanjem sadržaja ove datoteke se može izbeći kucanje komandi za svako podešavanje koje želimo da promenimo. Dodatne informacije o |git config| komandi kao i neka zgodna podešavanja se mogu naći u dodatku *.

\subsection{git init, git clone}
\label{subsec:git_init}
Git repozitorijum se može inicijalizovati u praznom direktorijumu, ali i u direktorijumu u kome se već radi na projektu. Za ovo se koristi komanda |git init|. Time će biti napravljen direktorijum \textit{.git} koji sadrži meta podatke repozitorijuma i pored toga sve ostalo ostaje nepromenjeno. Primer inicijalizacije repozitorijuma u novonapravljenom direktorijumu moze se videti na slici XXX.

U velikoj većini slučajeva, |git init| se poziva samo jednom, za pravljanje centralnog repozitorijuma. Kasnije, svi oni koji učestvuju u razvoju će taj repozitorijum kopirati na lokalne mašine korišćenjem komande:
\begin{lstlisting}[language=bash]
git clone <repozitorijum>
\end{lstlisting}
\noindent
gde |<repozitorijum>| treba zameniti adresom ogriginalnog repozitorijuma. Lokalni repozitorijum ima svoju istoriju izmena, svoje fajlove i potpuno je nezavisno okruženje od originalnog repozitorijuma. 


\subsection{git diff}
\label{subsec:git_diff}


\subsection{git add}
\label{subsec:git_add}
Nakon pravljenja izmena u radnoj verziji koje želite da upišete u repozitorijum, potrebno je priprema tih datoteka, odnosno dodavanje datoteka u Staging Area. Priprema datoteka za komitovanje se vrši komandom |git add| koja može bit pozvana više puta pre komitovanja, za istu ili neku drugu datoteku.



\subsection{git status}
\label{subsec:git_status}
Provera koje su to datoteke spremne za komitovanje (u Stagig Area) može se uraditi komandom |git status|. Pored toga, korišćenjem iste komande prikazuje se koje su datoteke promenjene u odnosu na repozitorijum, ali nisu spremne za komitovanje, kao i datoteke koje ne postoje u repozitorijumu, odnosno sistem za kontrolu verzija ih i dalje ne prati.


\subsection{git commit}
\label{subsec:git_commit}
Upisivanje promena u repozitorijum, odnosno komitovanje, zahteva od korisnika i poruku kojom opisuje promene. Pozivom |git commit| komande otvara se podrazumevani tekst editor gde je potrebno uneti opis, a nakon toga se komit čuva u repozitorijumu. U primeru na slici XXX, -m opcijom i navođenjem poruke nakon nje smo naznačili smo da želimo da koristimo tu poruku kao opis i tada se ne otvara tekst editor.


git help

git commit

git push

git pull

\section{Rad sa granama}
\label{sec:grane}

Ovde pišem tekst. 
Ovde pišem tekst. 
Ovde pišem tekst. 
Ovde pišem tekst. 

\subsection{Pravljenje i praćenje grana}
\label{subsec:pravljenje_grana}

\subsection{Rebase}
\label{subsec:rebase}

\subsection{Merge}
\label{subsec:merge}
Ovde pišem tekst. 
Ovde pišem tekst. 
Ovde pišem tekst. 
Ovde pišem tekst. 
Ovde pišem tekst. 
Ovde pišem tekst. 

\section{Rad sa udaljenim repozitorijumima}
\label{sec:udaljeni_repozitorijumi}

Ovde pišem tekst. 
Ovde pišem tekst. 
Ovde pišem tekst. 
Ovde pišem tekst. 
Ovde pišem tekst. 

\section{Zaključak}
\label{sec:zakljucak}

Ovde pišem zaključak. 
Ovde pišem zaključak. 
Ovde pišem zaključak. 
Ovde pišem zaključak. 
Ovde pišem zaključak. 
Ovde pišem zaključak. 
Ovde pišem zaključak. 
Ovde pišem zaključak. 
Ovde pišem zaključak. 
Ovde pišem zaključak. 
Ovde pišem zaključak. 
Ovde pišem zaključak. 


\addcontentsline{toc}{section}{Literatura}
\appendix
\bibliography{seminarski} 
\bibliographystyle{plain}

\appendix
\section{Dodatak}
Ovde pišem dodatne stvari, ukoliko za time ima potrebe.
Ovde pišem dodatne stvari, ukoliko za time ima potrebe.
Ovde pišem dodatne stvari, ukoliko za time ima potrebe.
Ovde pišem dodatne stvari, ukoliko za time ima potrebe.
Ovde pišem dodatne stvari, ukoliko za time ima potrebe.


\end{document}
